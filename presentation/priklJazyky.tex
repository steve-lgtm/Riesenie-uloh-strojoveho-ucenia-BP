\subsection{Vyjadrenie funkcie v informatike}
\frame{
	\frametitle{Jazyky}
	\begin{block}{Funkcia \alert{$vyp(n,v)$} v paradigmách programovania}
	\alert{Jazyk C}\hspace{3.7cm}\alert{Jazyk Ocaml}	
		\begin{columns}
			\begin{column}{0.37\textwidth}
				\footnotesize
				\begin{semiverbatim}

					int vypocet(int n) \{
	
					\hspace{0.3cm}int v =1;
	
					\hspace{0.3cm}while (n > 0) \{
	
					\hspace{0.57cm}v = v * (n \% 10);
	
					\hspace{0.57cm}n = n / 10;
	
					\hspace{0.3cm}\}
	
					\hspace{0.3cm}return v;
			
					\}
	
		\end{semiverbatim}

			\end{column}
			\begin{column}{0.64\textwidth}
				\footnotesize
				\begin{semiverbatim}

					let rec vypocet n v =
	
					\hspace{0.3cm}if n == 0 then
	
					\hspace{0.57cm}v

					\hspace{0.3cm}else
		
					\hspace{0.45cm}vypocet (n/10) (v * (n mod 10));;

				\end{semiverbatim}
			\end{column}
		\end{columns}
			
	\end{block}
} 

% Tu je demonštrované, že obidva modely vypočítateľnosti funkcie definované na začiatku 20. storočia p. Turingom (turingovho stroja) a p. Churchom (lambda kalkulu) môžu definovať rovnakú triedu vypočítateľných funkcií.