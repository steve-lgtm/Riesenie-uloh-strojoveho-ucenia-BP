\subsection{Fukncia vyp}
\frame{
	\frametitle{Funkcia v kategórii}
	\begin{block}{Fukncia $vyp(n,v)$ z pohľadu Teórie kategórií}
		
		\begin{columns}
			\begin{column}{0.4\textwidth}
			Tvar klasifikujúcej kategórie:
			\vspace{2.0cm}
			
			Signatúra \alert{$\Sigma=(T,F)$} \\
		
			\end{column}
			
			\begin{column}{0.6\textwidth}
					\structure{\begin{center}
						\pgfuseimage{prikl1}
					\end{center}}
			\end{column}
	\end{columns}	

	    	\structure{\begin{tabular}{l}
						$\alert{T}=\{int\}$ \\
						$\alert{F}=\{\dots vyp: int \rightarrow int \rightarrow int \dots\}$ \\
		    \end{tabular}}

			\vspace{0.3cm}

		Term:
		
		\alert{$\scriptstyle{n:int,~v:int~\vdash~if~n=0~then~v~else~vyp~(n~div~10)~(v~mult~(n~mod~10)):int}$}
		
	\end{block}
} 

% \scriptstyle{n:int, v:int, \vdash if(n=0,v,f(div(n,10),mult(v,mod(n,10)))):int}
%			\scriptstyle{n:int,~v:int \vdash if~n=0~then~v~else~vyp~(n~div~10)~(v~mult~(n~mod~10))}