\chapter{Formátovanie dokumentu}

\section{Štruktúra projektu}

Projekt záverečnej práce má nasledovnú štruktúru:

\begin{verbatim}
.
|-- bibliography.bib
|-- CHANGELOG.md
|-- chapters
|   |-- formatovanie.tex
|   |-- instalacia.tex
|   `-- struktura.tex
|-- figures
|   |-- foto.png
|   `-- tugboat.png
|-- kpithesis.cls
`-- thesis.tex
\end{verbatim}

Význam jednotlivých súborov a priečinkov je nasledovný:

\begin{itemize}
    \item súbor {\tt \bf{bibliography.bib}} obsahuje zoznam literatúry vo formáte \emph{BibTeX}
    \item priečinok {\tt \bf{chapters/}} obsahuje {\tt .tex} súbory reprezentujúce samostatné kapitoly záverečnej práce
    \item priečinok {\tt \bf{figures/}} obsahuje zoznam obrázkov, ktoré boli v práci použité
    \item v súbore {\tt \bf{kpithesis.cls}} sa nachádza samotná šablóna \emph{kpithesis}
    \item súbor {\tt \bf{thesis.tex}} predstavuje hlavný súbor záverečnej práce
\end{itemize}


\section{Vkladanie obrázkov}

Všetky obrázky, ktoré budete chcieť v dokumente použiť, ukladajte do priečinku {\tt figures/}. Následne obrázok vložte do dokumentu pomocou prostredia \verb!figure! pomocou príkazu \verb!\includegraphics! bez uvedenia jeho prípony. Napríklad takto:

\begin{minted}{latex}
\begin{figure}[!ht]
    \centering
    \includegraphics[width=\textwidth]{figures/tugboat}
    \caption{\LaTeX{} Friendly Zone \label{o:latex_friendly_zone}}
\end{figure}
\end{minted}

Výsledok tohto fragmentu kódu sa nachádza na obrázku \ref{o:latex_friendly_zone}.

\begin{figure}[!ht]
    \centering
    \includegraphics[width=\textwidth]{figures/tugboat}
    \caption{\LaTeX{} Friendly Zone \label{o:latex_friendly_zone}}
\end{figure}

Ak chcete obrázok vložiť do dokumentu otočený o $90$, môžete použiť voľbu {\tt angle=90}, ktorú poskytuje balík {\tt graphicx}:

\begin{minted}{latex}
\begin{figure}[!ht]
    \centering
    \includegraphics[angle=90,width=\textwidth]{figures/tugboat}
    \caption{\LaTeX{} Friendly Zone \label{o:latex_friendly_zone_90}}
\end{figure}
\end{minted}

Výsledkom tejto úpravy je obrázok \ref{o:latex_friendly_zone_90}.

\begin{figure}[!ht]
    \centering
    \includegraphics[angle=90,width=\textwidth]{figures/tugboat}
    \caption{\LaTeX{} Friendly Zone \label{o:latex_friendly_zone_90}}
\end{figure}

V prípade, ak chcete otočiť obrázok spolu s popiskom, použite balíček {\tt rotating}, ktorý poskytuje prostredie {\tt sidewaysfigure} nasledovným spôsobom:

\begin{minted}{latex}
\begin{sidewaysfigure}
    \centering
    \includegraphics[width=.7\textheight]{figures/tugboat}
    \caption{\LaTeX{} Friendly Zone \label{o:latex_friendly_zone_rotating}}
\end{sidewaysfigure}
\end{minted}

Výsledok tohto fragmentu kódu sa nachádza na obrázku \ref{o:latex_friendly_zone_rotating}.

\begin{sidewaysfigure}
    \centering
    \includegraphics[width=.7\textheight]{figures/tugboat}
    \caption{\LaTeX{} Friendly Zone \label{o:latex_friendly_zone_rotating}}
\end{sidewaysfigure}

\section{Vkladanie tabuliek}
