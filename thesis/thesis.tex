%% -----------------------------------------------------------------
%% This file uses UTF-8 encoding
%%
%% For compilation use following command:
%% latexmk -pdf -pvc -bibtex --shell-escape thesis
%% -----------------------------------------------------------------
\documentclass{kithesis}

% balicky
\usepackage[slovak]{babel}

% syntax higlighting and it's configuration
\usepackage{minted}  
\renewcommand\listoflistingscaption{Zoznam zdrojových kódov}
\renewcommand\listingscaption{Výpis}
\newminted{latex}{frame=lines,framerule=2pt}
\usemintedstyle{tango}

\usepackage{blindtext}  % lorem ipsum
\usepackage{nicefrac}  % pekne zlomky
\usepackage{rotating}  % umoznuje otacat obrazky spolu s popiskami
\usepackage{csquotes}

% premenne
\title{Moja diplomová práca\\
	s názvom cez dva riadky}

\author{Janko Hraško}
\supervisor{Leslie Lamport} %veduciprace
\consultant{Donald E. Knuth} %konzultant
%\college{Žilinská univerzita}{Žilina} %univerzita
%\faculty{Fakulta elektrotechniky a informatiky} %fakulta
\department{Katedra počítačov a informatiky}{KPI} %katedra
%\thesis{Diplomová práca} %typprace
\submissiondate{13}{5}{2016}
%\fieldofstudy{9.2.1 Informatika}
%\studyprogramme{Informatika}
%\city{Košice} %mesto
\keywords{\LaTeX, programming, typesetting}{\LaTeX, programovanie, sadzba textu}
%\declaration{som nepodvadzal}

\abstract{%
    anglicky \LaTeX \blindtext
}{%
    slovensky \blindtext
}

\acknowledgment{Na tomto mieste by som rád poďakoval svojmu vedúcemu práce za jeho čas a odborné vedenie počas riešenia mojej záverečnej práce.

Rovnako by som sa rád poďakoval svojim rodičom a priateľom za ich podporu a povzbudzovanie počas celého môjho štúdia.
    
V neposlednom rade by som sa rád poďakoval pánom {\it Donaldovi E. Knuthovi} a {\it Leslie Lamportovi} za ich typografický systém \LaTeX, s ktorým som strávil množstvo nezabudnuteľných večerov.}

\addbibresource{thesis.bib}

% if you want to work only on selected chapters
%\includeonly{chapters/struktura,chapters/formatovanie}
%%%%%%%%%%%%%%%%%%%%%%%%%%%%%%%%%%%%%%%%%%%%%%%%%%%%%%%%%%%%%%%%%%%%%%%%%%%%%%%%
\begin{document}

%% Title page, abstract, declaration etc.:
\frontmatter{}

%% List of code listings
\listoflistings

% list of acronyms
\thispagestyle{empty}
% Acronyms
% ========
%
% An acronym is a word formed from the initial letters in a phrase. 
%
% Acronym Definition Exapmle:
% ---------------------------
% \newacronym{gcd}{GCD}{Greatest Common Divisor}
% \newacronym{dry}{DRY}{Don't Repeat Yourself}
%
% Usage:
% ------
% You can use these three options:
% 
% \acrlong{}  
%   Displays the phrase which the acronyms stands for. Put the label of the acronym inside the braces. In the example, \acrlong{gcd} prints Greatest Common Divisor. 
%
% \acrshort{} 
%   Prints the acronym whose label is passed as parameter. For instance, \acrshort{gcd} renders as GCD. 
%
% \acrfull{ } 
%   Prints both, the acronym and its definition. In the example the output of \acrfull{dry} is Don't Repeat Yourself (DRY). 
% 
% For more information see:
% -------------------------
% * https://www.sharelatex.com/learn/Glossaries 
% * https://en.wikibooks.org/wiki/LaTeX/Glossary
%


\newacronym{gcd}{GCD}{Greatest Common Divisor}
\newacronym{lcm}{LCM}{Least Common Multiple}

\printglossary[type=\acronymtype,title={\acrlistname}]
\newpage

\pagenumbering{arabic}


%
% !TEX root = ../thesis.tex

\chapter{Typografický systém \LaTeX}
\label{ch:instalacia}

\section{Ako začať s \LaTeX{}om}

Pre zvládnutie tohto jazyka neexistuje lepší spôsob, ako v ňom proste začať dokumenty rovno písať. Pre zvládnutie základov odporúčame použiť voľne dostupnú publikáciu \emph{The Not So Short Introduction to \LaTeX} \cite{lshort}, ktorú do slovenčiny preložili \emph{Ján Buša st.} a \emph{Ján Buša ml.} \cite{lshortsk}.

Môžete rovnako siahnuť aj po originálnych československých zdrojoch. Voľne dostupná, stručná a zrozumiteľná publikácia je \emph{\LaTeX pro pragmatiky} \cite{satrapa2011} od \emph{Pavla Satrapu}. Pre dôkladnejšie zoznámenie sa s prácou v jazyku poslúži kniha \emph{\LaTeX pro začátečníky} \cite{rybicka2003} od \emph{Jiřího Rybičku}. 


\section{Inštalácia v OS Windows}

Ak pracujete v \emph{OS Windows}, stiahnite si distribúciu \LaTeX-u s názvom \emph{TeX Live} zo stránky \url{https://www.tug.org/texlive/}. Inštaláciu distribúcie \emph{MikTeX} neodporúčame.


\section{Inštalácia v OS Linux}

Ak používate distribúciu \emph{Fedora 23}, pre používanie šablóny budete potrebovať nainštalovať nasledujúce balíčky:

\begin{minted}{bash}
$ sudo dnf install texlive-bibtopic texlive-cslatex \
                 texlive-collection-latex \
                 texlive-collection-fontsrecommended \
                 texlive-cite latexmk texlive-textcase \
                 texlive-engrec texlive-parskip \
                 texlive-minted \
                 texlive-europecv \
                 texlive-hyphen-english texlive-hyphen-slovak \
                 texlive-titlesec
\end{minted}

\section{Inštalácia v OSX}

TODO: Nakoľko nedisponujeme týmto systémom, ak má niekto z vás skúsenosti s inštaláciou systému \LaTeX na tento operačný systém, ozvite sa ;)


\section{Ďaľšie odporúčané balíčky}
\begin{itemize}
    \item \emph{minted}\footnote{\url{https://www.ctan.org/pkg/minted}} - zvýrazňovač zdrojového kódu (syntax highlighter) pre \LaTeX 
    \item \emph{europecv}\footnote{\url{https://www.ctan.org/pkg/europecv}} - šablóna pre písanie životopisov vo formáte EuroPass 
    \item \emph{PGF/TikZ}\footnote{\url{http://www.ctan.org/pkg/pgf}} - dvojica jazykov, pomocou ktorých je možné vytvárať vektorovú grafiku 
    \item \emph{rotating}\footnote{\url{https://www.ctan.org/pkg/rotating}} - umožňuje otáčať obrázky a tabuľky (spolu s ich popiskami)
\end{itemize}


\section{Generovanie \LaTeX dokumentov}

Pre priame spúšťanie z príkazového riadku odporúčame použiť príkaz {\tt latexmk}, ktorý slúži na zostavenie  \LaTeX dokumentov. Príklad použitia je nasledovný:

\begin{minted}{bash}
$ latexmk -pdf -bibtex -shell-escape thesis
\end{minted}

Ak nechcete spúšťať tento príkaz zakaždým po vykonaní zmien v zdrojových súboroch, pridajte programu {\tt latexmk} prepínač {\tt -pvc}, ktorý zabezpečí ich sledovanie a znovuzostavenie výstupu automaticky:

\begin{minted}{bash}
$ latexmk -pdf -bibtex -shell-escape -pvc thesis
\end{minted}

Ak budete chcieť vyčistiť vygenerované výstupy, stačí nástroj {\tt latexmk} spustiť s prepínačom {\tt -c}:

\begin{minted}{bash}
$ latexmk -c thesis
\end{minted}


\section{Nástroj vlna}

Význam tohto nástroja je zhrnutý v úvode jeho manuálovej stránky:

\begin{quote}
There exists a special Czech and Slovak typographical rule: you cannot leave the non-syllabic preposition on the end of one line and continue writting text on next line. For  example, you  cannot  write  down  the text "v lese" (in a forest) like "v<new-line>lese". The program vlna adds the asciitilde between  such  preposition  and  the next  word and removes the space(s) in this place.  It means, the program converts "v lese" to "v~lese". You  can  use  this  program  as  a  preporcessor  before  TeXing. Moreower,  you  can  set  another  sequence  to  store instead asciitilte (see the -x option).
\end{quote}

Takže ak záverečnú prácu píšete v slovenskom alebo českom jazyku, odporúčame nástroj vlna spustiť minimálne raz pred samotným odovzdaním. Samozrejme ho môžete spustiť vždy, keď svoju prácu budete posielať svojmu školiteľovi alebo konzultantovi na kontrolu.

\section{Štruktúra záverečnej práce}

Záverečná práca sa skladá z týchto častí:

\begin{enumerate}
    \item Predhovor
    \item Abstrakt
    \item Úvod práce
    \item Analytická časť práce
    \item Syntetická časť práce
    \item Vyhodnotenie
    \item Záver
	\item Prílohy
	\begin{itemize}
	    \item Životopis
	    \item Systémová príručka
	    \item Používateľská príručka
	\end{itemize}
\end{enumerate}

Záverečná práca musí obsahovať pôvodné myšlienky vytvorené autorom, nesmie byť len jednoduchým prerozprávaním známych faktov a postupov.

\subsection{Predhovor}

Predhovor v záverečnej práci nie je povinný. Ak je predhovor v práci uvedený, potom obsahuje dôvody pre
voľbu témy práce a pozadie realizácie práce.

\subsection{Abstrakt}

Abstrakt je stručný opis obsahu záverečnej práce. Z abstraktu musí byť čitateľovi zrejmé čo autor v práci
riešil (problém), ako to riešil (metódy), k čomu v práci dospel (výsledky) a aké sú prínosy jeho riešenia.

\subsection{Úvod práce}

Úvod práce stručne opisuje stanovený problém, kontext problému a motiváciu pre riešenie problému. Z úvodu by malo byť jasné, že stanovený problém doposiaľ nie je vyriešený a má zmysel ho riešiť. Súčasťou úvodu práce je formulácia úlohy (samostatná kapitola), v ktorej sú jasne stanovené ciele záverečnej práce na základe problému. V úvode neuvádzajte štruktúru práce, t.j. o čom je ktorá kapitola. Rozsah úvodu je minimálne 2 celé strany (vrátane formulácie úlohy). Jadro práce musí obsahovať analytickú, syntetickú a vyhodnocovaciu časť. Názvy jednotlivých kapitol a členenie jadra je ponechané na autora.
    
\subsection{Analytická časť práce}

Analytická časť záverečnej práce analyzuje existujúce podobné prístupy k riešeniu stanoveného problému. Autor práce musí uviesť v tejto časti existujúce prístupy a riešenia, pričom musí zaujať stanovisko k týmto prístupom a riešeniam a opísať ich výhody a nedostatky. Prevažne v tejto časti autor používa odkazy na použité zdroje. Autor v analýze nepreberá odseky z cudzích prác ale uvádza prevažne vlastné postoje podložené odkazmi na literatúru. Je odporúčané aby bola analýza podporená aj experimentmi ak to umožňuje téma práce (napr. vyskúšam softvér). Analytická časť tvorí zvyčajne \nicefrac{1}{4} jadra práce.

\subsection{Syntetická časť práce}

Syntetická časť opisuje metódy použité na syntézu riešenia a opisuje syntézu samotného riešenia (zvyčajne je to návrh/implementácia softvérového resp. hardvérového riešenia), pričom sa opiera o závery analytickej časti práce. Syntetická časť tvorí zvyčajne \nicefrac{1}{2} jadra práce.

\subsection{Vyhodnotenie}

Vyhodnocovacia časť je kľúčovou časťou záverečnej práce. Tato časť obsahuje vyhodnotenie navrhnutého (vytvoreného) riešenia. Uprednostňované je objektívne vyhodnotenie výsledkov práce, ktoré sa opiera o meranie a štatistické metódy, prípadne matematické dôkazy. V prípade nameraných hodnôt musí autor opísať metódu merania, priebeh merania, výsledky a interpretáciu výsledkov v kontexte riešeného problému a stanovených cieľov. Na základe vyhodnotenia riešenia autor opíše prínosy svojej práce. Vyhodnocovacia časť tvorí zvyčajne \nicefrac{1}{4} jadra práce.

\subsection{Záver}

Záver práce obsahuje zhrnutie výsledkov práce s jasným opisom prínosov a pôvodných (vlastných) výsledkov autora a vyhodnotenie splnenia stanovených cieľov. Je to stručné zhrnutie informácií uvedených v záverečnej práci. Záver by nemal obsahovať nové informácie. V závere by mal tiež autor poukázať na prípadné otvorené otázky, ktoré sú nad rámec rozsahu práce a mal by odporučiť ďalšie aktivity na pokračovanie pri riešení problému. Rozsah záveru je minimálne 1 celá strana.



\chapter{Formátovanie dokumentu}

\section{Štruktúra projektu}

Projekt záverečnej práce má nasledovnú štruktúru:

\begin{verbatim}
.
|-- bibliography.bib
|-- CHANGELOG.md
|-- chapters
|   |-- formatovanie.tex
|   |-- instalacia.tex
|   `-- struktura.tex
|-- figures
|   |-- foto.png
|   `-- tugboat.png
|-- kpithesis.cls
`-- thesis.tex
\end{verbatim}

Význam jednotlivých súborov a priečinkov je nasledovný:

\begin{itemize}
    \item súbor {\tt \bf{bibliography.bib}} obsahuje zoznam literatúry vo formáte \emph{BibTeX}
    \item priečinok {\tt \bf{chapters/}} obsahuje {\tt .tex} súbory reprezentujúce samostatné kapitoly záverečnej práce
    \item priečinok {\tt \bf{figures/}} obsahuje zoznam obrázkov, ktoré boli v práci použité
    \item v súbore {\tt \bf{kpithesis.cls}} sa nachádza samotná šablóna \emph{kpithesis}
    \item súbor {\tt \bf{thesis.tex}} predstavuje hlavný súbor záverečnej práce
\end{itemize}


\section{Vkladanie obrázkov}

Všetky obrázky, ktoré budete chcieť v dokumente použiť, ukladajte do priečinku {\tt figures/}. Následne obrázok vložte do dokumentu pomocou prostredia \verb!figure! pomocou príkazu \verb!\includegraphics! bez uvedenia jeho prípony. Napríklad takto:

\begin{minted}{latex}
\begin{figure}[!ht]
    \centering
    \includegraphics[width=\textwidth]{figures/tugboat}
    \caption{\LaTeX{} Friendly Zone \label{o:latex_friendly_zone}}
\end{figure}
\end{minted}

Výsledok tohto fragmentu kódu sa nachádza na obrázku \ref{o:latex_friendly_zone}.

\begin{figure}[!ht]
    \centering
    \includegraphics[width=\textwidth]{figures/tugboat}
    \caption{\LaTeX{} Friendly Zone \label{o:latex_friendly_zone}}
\end{figure}

Ak chcete obrázok vložiť do dokumentu otočený o $90$, môžete použiť voľbu {\tt angle=90}, ktorú poskytuje balík {\tt graphicx}:

\begin{minted}{latex}
\begin{figure}[!ht]
    \centering
    \includegraphics[angle=90,width=\textwidth]{figures/tugboat}
    \caption{\LaTeX{} Friendly Zone \label{o:latex_friendly_zone_90}}
\end{figure}
\end{minted}

Výsledkom tejto úpravy je obrázok \ref{o:latex_friendly_zone_90}.

\begin{figure}[!ht]
    \centering
    \includegraphics[angle=90,width=\textwidth]{figures/tugboat}
    \caption{\LaTeX{} Friendly Zone \label{o:latex_friendly_zone_90}}
\end{figure}

V prípade, ak chcete otočiť obrázok spolu s popiskom, použite balíček {\tt rotating}, ktorý poskytuje prostredie {\tt sidewaysfigure} nasledovným spôsobom:

\begin{minted}{latex}
\begin{sidewaysfigure}
    \centering
    \includegraphics[width=.7\textheight]{figures/tugboat}
    \caption{\LaTeX{} Friendly Zone \label{o:latex_friendly_zone_rotating}}
\end{sidewaysfigure}
\end{minted}

Výsledok tohto fragmentu kódu sa nachádza na obrázku \ref{o:latex_friendly_zone_rotating}.

\begin{sidewaysfigure}
    \centering
    \includegraphics[width=.7\textheight]{figures/tugboat}
    \caption{\LaTeX{} Friendly Zone \label{o:latex_friendly_zone_rotating}}
\end{sidewaysfigure}

\section{Vkladanie tabuliek}

%%
%% !TEX root = ../thesis.tex

\chapter{Formulácia úlohy}

Text záverečnej práce musí obsahovať\/ kapitolu s~formuláciou úlohy resp. úloh riešených v~rámci záverečnej práce. V~tejto časti autor rozvedie spôsob, akým budú riešené úlohy a~tézy formulované v~zadaní práce. Taktiež uvedie prehľad podmienok riešenia.

%%
%\chapter{Analytická časť práce}

% lorem ipsum
\section{Lorem ipsum}
\Blindtext
\blinditemize

\section{Aliquam eu malesuada urna}
\blindtext
\begin{itemize}
    \item v knihe \cite{book} autor prezentuje naozaj odvážne myšlienky
    \item nemenej zujímavé výsledky publikuje ďalší autor v článku \cite{article} 
    \item v konferenčnom príspevku \cite{conference} sú uvedené tiež zauímavé veci
\end{itemize}

Given a set of numbers, there are elementary methods to compute its \acrlong{gcd}, which is abbreviated \acrshort{gcd}. This process is similar to that used for the \acrfull{lcm}.

\subsection{Donec vehicula consequat}
\blindtext

\subsection{Nullam in mauris consectetur}
\blindtext
\blindenumerate

\subsection{Vestibulum tristique elementum varius}
\blindtext

\section{Phasellus id pretium neque}
\Blindtext

%%
%\include{jadroprace}
%%
%% !TEX root = ../thesis.tex

\chapter{Záver}

% lorem ipsum
\Blindtext

%%
%\include{literatura}
%\bibliographystyle{dcu}
%\bibliography{bibliography}{}
\printbibliography[title={Literatúra}]
%%
%\chapter*{Zoznam príloh}

\addcontentsline{toc}{chapter}{\numberline{}Zoznam príloh}
\thispagestyle{empty}

\begin{description}
	\item[Príloha A] Lorem ipsum text
    \item[Príloha B] CD médium -- záverečná práca v~elektronickej podobe,
    \item[Príloha C] Používateľská príručka
    \item[Príloha D] Systémová príručka
\end{description}


%%
%% !TEX root = ../thesis.tex

\chapter{Karel Language Reference}

\section*{Karel's Primitives}

\begin{itemize}
    \item \verb|void movek()| - Moves \textit{Karel} one intersection forward.
    \item \verb|void turn_left()| - Pivots \textit{Karel} $90$ degrees left.
    \item \verb|void pick_beeper()| - Takes a beeper from the current intersection and puts it in the beeper bag.
    \item \verb|void put_beeper()| - Takes a beeper from the beeper bag and puts it at the current intersection.
    \item \verb|void turn_on(char* path)| - Turns \textit{Karel} on.
    \item \verb|void turn_off()| - Turns \textit{Karel} off.
\end{itemize}


\section*{Karel's Sensors}

\begin{itemize}
    \item \verb|int front_is_clear()| - Returns \texttt{1} if there is no wall directly in front of \textit{Karel}. \texttt{0} if there is.
    \item \verb|int right_is_clear()| - Returns \texttt{1} if there is no wall immediately to \textit{Karel}'s right. \texttt{0} if there is.
    \item \verb|int beepers_present()| - Returns \texttt{1} if \textit{Karel} is standing at an intersection that has a beeper. \texttt{0} otherwise.
    \item \verb|int facing_north()| - Returns \texttt{1} if \textit{Karel} is facing north. \texttt{0} otherwise.
    \item \verb|int beepers_in_bag()| - Returns \texttt{1} if there is at least one beeper in \textit{Karel}'s beeper bag. \texttt{0} if the beeper bag is empty.
\end{itemize}


\section*{Misc Functions}

\begin{itemize}
    \item \verb|void set_step_delay(int)| - Sets delay of one \textit{Karel}'s step in miliseconds.
    \item \verb|loop(int)| - Repeats \textit{Karel}'s instruction in a loop.
\end{itemize}

%%
%\include{prilohab}
%%
%\include{prilohac}
%
% zivotopis autora
%\curriculumvitae\protect
%\label{page:posledna}
%Táto časť\/ je nepovinná. Autor tu môže uviesť\/ svoje biografické
%údaje, údaje o~záujmoch, účasti na~projektoch, účasti na~súťažiach,
%získané ocenenia, zahraničné pobyty na~praxi, domácu prax, publikácie
%a~pod.

\end{document}
