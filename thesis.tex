\documentclass[a4paper]{tukediphc}
%% -----------------------------------------------------------------
%% tento subor ma kodovanie utf-8
%%
%% na kompilaciu pouzivajte format pdfcslatex 
%%
%% vytvorene distribuciou texlive 2009-14, OS GNU/Linux
%% vytvorene distribuciou TeXLive 2010, OS Win XP
%% 
%% -----------------------------------------------------------------
% pouzite balicky
\usepackage[slovak]{babel}


\usepackage[utf8]{inputenc}
\usepackage[T1]{fontenc}

\usepackage{latexsym}

\usepackage[pdftex]{hyperref}

% popisky ku obrazkom a tabulkam
\def\figurename{Obrázok}
\def\tabname{Tabuľka}
\def\refname{Zoznam použitej literatúry}


%% Citácie
%% Cislovane citovanie
\usepackage[numbers]{natbib}
%%
%% Citovanie podľa mena autora a roku
% \usepackage{natbib} \citestyle{chicago}

\usepackage{lmodern,textcase}
\usepackage{dcolumn} % zarovnanie cisiel v tabulke podla des. ciarky
\usepackage{hhline}
\usepackage{amsmath}
\usepackage{nicefrac} % pekne zlomky
\usepackage{upgreek} % napr. $\upmu\mathrm{m}$ pre mikrometer ...
\usepackage[final]{showkeys}%color%notref%notcite%final
\usepackage[slovak,noprefix]{nomencl}
\makeglossary % prikaz na vytvorenie suboru .glo
\usepackage{parskip}% 'zhusti' polozky obsahu

%% Obrázky a grafika
%\usepackage[dvips]{graphicx}
\usepackage[pdftex]{graphicx}
\DeclareGraphicsExtensions{.pdf,.png,.jpg,.mps}
\graphicspath{{figures/}} % priecinok na obrazky
%%

%% Ďalšie (špeciálne) balíčky
%% Prostredie pre zdrojové kódy
%\usepackage[slovak,linesnumbered,lined,algoruled]{algorithm2e}
\usepackage{minted}
%% Prostredie pre tvorbu vlastnej grafiky, príklady na http://www.texample.net/tikz/examples/
\usepackage{tikz}



% -----------------------------------------------------------------
%% tlač !!!
% \usepackage[pdftex,unicode=true,bookmarksnumbered=true,
% bookmarksopen=true,pdfmenubar=true,pdfview=Fit,linktocpage=true,
% pageanchor=true,bookmarkstype=toc,pdfpagemode=UseOutlines,
% pdfstartpage=1]{hyperref}

% \hypersetup{%
% baseurl={http://www.tuke.sk/sevcovic},
% pdfcreator={pdfcsLaTeX},
% pdfkeywords={Optimaliz\'acia, diplomov\'a pr\'aca, p\'isanie},
% pdftitle={Optimalizácia písania diplomových prác},
% pdfauthor={J\'an Zelen\'y},
% pdfsubject={Bakalárska práca}
% } 


% pre tlac treba hypersetup zakomentovat
\hypersetup{
   unicode,
   backref,
   pdftoolbar=true,
   pdfmenubar=true,
   pdfwindowui=true,
   bookmarksopenlevel={0},
   bookmarksnumbered,
   bookmarksopen,
   pdfhighlight={/P},
   colorlinks,
   citecolor=magenta,
   bookmarksnumbered,
   pdfkeywords = {\LaTeX},
   pdfcreator = {\LaTeX\ with package \flqq hyperref\frqq},
   pdftitle={Optimaliz\'acia, diplomov\'a pr\'aca, p\'isanie},
   pdfauthor={Janko Hra\v{s}ko},
   pdfsubject={Bakal\'arska pr\'aca}, % Diplomov\'a, Dizerta\v{c}n\'a
   baseurl={http://www.tuke.sk}
}



%% nehodiace zakomentujte !
%\dippraca{Diplomová práca}
\bakpraca{Bakalárska práca}
%%
\nazov{Optimalizácia písania diplomových prác}
%% ked praca nema 'podnazov' zakomentujte nasledujuci riadok
%% alebo polozku nechajte prazdnu
\podnazov{}
\autor{Ján Zelenka-Košický}
\veduciprace{doc.~Ing.~Vojtech~Čierny, CSc.}
\konzultanta{Ing.~Matej~Biely, PhD.}
\konzultantb{RNDr.~Marián~Čierny, DrSc.}
\univerzita{Technická univerzita v~Košiciach}
\fakulta{Fakulta elektrotechniky a informatiky}
\skratkafakulty{FEI}
\katedra{Katedra umelej inteligencie}
\skratkakatedry{KUI}
\odbor{Experimentálna fyzika (pozri zadávací list)}
\specializacia{Fyzika nízkych teplôt}
\abstrakt{Abstrakt je povinnou súčasťou každej práce. Je výstižnou charakteristikou obsahu dokumentu. Nevyjadruje hodnotiace stanovisko autora. Má byť\/ taký informatívny, ako to povoľuje podstata práce. Text abstraktu sa píše ako jeden odstavec. Abstrakt neobsahuje odkazy na samotný text práce. Mal by mať\/ rozsah 250 až 500 slov. Pri štylizácii sa používajú celé vety, slovesá v činnom rode a tretej osobe. Používa sa odborná terminológia, menej zvyčajné termíny, skratky a~symboly sa pri prvom výskyte v texte definujú.}
\klucoveslova{Optimalizácia, diplomová práca, písanie}
\abstrakte{Text abstraktu v~svetovom jazyku je potrebný pre integráciu
do medzinárodných informačných systémov. Ak nie je možné cudzojazyčnú
verziu abstraktu umiestniť na jednej strane so slovenským abstraktom,
je potrebné umiestniť ju na samostatnú stranu (cudzojazyčný abstrakt
nemožno deliť a~uvádzať na dvoch strabách).}
\keywords{Optimization, diploma, writing}
\datumodovzdania{11. 4. 2013}
\mesto{Košice}

\begin{document}
\renewcommand\theHfigure{\theHsection.\arabic{figure}}
\renewcommand\theHtable{\theHsection.\arabic{table}}
\bibliographystyle{dcu}

\prvastrana

\titulnastrana

\errata % zaciatok erraty
Ak je potrebné, autor na tomto mieste opraví chyby, ktoré našiel po
vytlačení práce. Opravy sa uvádzajú takým písmom, akým je napísaná
práca. Ak zistíme chyby až po vytlačení a zviazaní práce, napíšeme
erráta na samostatný lístok, ktorý vložíme na toto miesto. Najlepšie je
lístok prilepiť\/ \citep{kat}.

Forma:

\tabcolsep=10pt
\begin{table}[!hb]
	\centering
	\begin{tabular}{|c|c|c|c|}\hline
Strana & Riadok & Chybne & Správne \\\hline\hline
12 & 6 & publikácia & prezentácia \\\hline
22 & 23 & internet & intranet \\\hline
& & & \\\hline
& & & \\\hline
	\end{tabular}
\end{table}
\kerrata % koniec erraty

\abstraktsk % abstrakt v SK 

\abstrakteng % abstrakt v ENG

\kabstrakt % koniec abstraktov, nova strana

% Na tomto mieste bude vložené zadanie diplomovej práce
\zadanieprace

\cestnevyhlasenie
% Niektorí autori metodických príručiek o~záverečných prácach sa
% nazdávajú, že takéto vyhlásenie je zbytočné, nakoľko povinnosť
% vypracovať záverečnú prácu samostatne, vyplýva študentovi zo zákona a
% na autora práce sa vzťahuje autorský zákon.

\podakovanie
Na tomto mieste môže byť\/ vyjadrenie poďakovania napr. vedúcemu
diplomovej práce, resp. konzultantom, za pripomienky a~odbornú pomoc
pri vypracovaní diplomovej práce.

Na tomto mieste môže byť\/ vyjadrenie poďakovania napr. vedúcemu
diplomovej práce, respektíve konzultantom, za pripomienky a~odbornú
pomoc pri vypracovaní diplomovej práce.

Na tomto mieste môže byť\/ vyjadrenie poďakovania napr. vedúcemu
diplomovej práce alebo konzultantom za pripomienky a~odbornú pomoc pri
vypracovaní diplomovej práce.
\kpodakovania

\predhovor
Predhovor je povinnou náležitosťou záverečnej práce, pozri
\citep{gonda}. V~predhovore autor uvedie základné charakteristiky
svojej záverečnej práce a~okolnosti jej vzniku. Vysvetlí dôvody, ktoré
ho viedli k~voľbe témy, cieľ a~účel práce a~stručne informuje
o~hlavných metódach, ktoré pri spracovaní záverečnej práce použil.
\kpredhovoru

\thispagestyle{empty}
\tableofcontents
\newpage

\thispagestyle{empty}
%\addcontentsline{toc}{section}{\numberline{}Zoznam obrázkov}
\listoffigures
\newpage

\thispagestyle{empty}
%\addcontentsline{toc}{section}{\numberline{}Zoznam tabuliek}
\listoftables
\newpage

\thispagestyle{empty}
%\addcontentsline{toc}{section}{\numberline{}Zoznam symbolov a
%skratiek}
\printglossary % vlozenie zoznamu skratiek a symbolov
\newpage

%\addcontentsline{toc}{section}{\numberline{}Slovník termínov}
\slovnikterminov

% odsadzovanie prveho riadku odstavca
\setlength{\parindent}{1cm}
% medzera medzi odstavcami
\setlength{\parskip}{1ex plus 0.5ex minus 0.2ex}

\begin{description}
	\item[Dizertácia] je rozsiahla vedecká rozprava, v~ktorej sa na
základe vedeckého výskumu a~s~použitím (využitím) bohatého dokladového
materiálu  ako i~vedeckých metód rieši zložitý odborný problém.
	\item[Font] je súbor, obsahujúci predpisy na zobrazenie textu
v~danom písme, napr. na tlačiarni. To čo vidíme je písmo; font je súbor
a~nevidíme ho.
	\item[Kritika] je odborne vyhrotený, prísny pohľad na hodnotenú
vec. Medzi recenziou a kritikou je taký pomer ako medzi diskusiou a
polemikou. Pri kritike treba prísnosť\/ chápať\/ v~tom zmysle, že sa
v~nej okrem iného navrhuje, ako hodnotené dielo skvalitniť\/.
	\item[Meter (m)] je vzdialenosť, ktorú svetlo vo vákuu prejde
za čas. interval~$\nicefrac{1}{299\,792\,458}$ sekundy.
	\item[Písmom] rozumieme vlastný vzhľad znakov.
	\item[Problém] termín používaný vo všeobecnom zmysle vo vzťahu
k~akejkoľvek duševnej aktivite, ktorá má nejaký rozoznateľný cieľ.
Samotný cieľ nemusí byť\/ v~dohľadne. Problémy možno charakterizovať\/
tromi rozmermi -- oblasťou, obtiažnosťou a veľkosťou.
	\item[Proces] je postupnonosť\/ či rad časovo usporiadaných
udalostí tak, že každá predchádzajúca udalosť\/ sa zúčastňuje na
determinácii nasledujúcej udalosti.
\end{description}

\kslovnikterminov
%
% !TEX root = ../thesis.tex

\chapter{Typografický systém \LaTeX}
\label{ch:instalacia}

\section{Ako začať s \LaTeX{}om}

Pre zvládnutie tohto jazyka neexistuje lepší spôsob, ako v ňom proste začať dokumenty rovno písať. Pre zvládnutie základov odporúčame použiť voľne dostupnú publikáciu \emph{The Not So Short Introduction to \LaTeX} \cite{lshort}, ktorú do slovenčiny preložili \emph{Ján Buša st.} a \emph{Ján Buša ml.} \cite{lshortsk}.

Môžete rovnako siahnuť aj po originálnych československých zdrojoch. Voľne dostupná, stručná a zrozumiteľná publikácia je \emph{\LaTeX pro pragmatiky} \cite{satrapa2011} od \emph{Pavla Satrapu}. Pre dôkladnejšie zoznámenie sa s prácou v jazyku poslúži kniha \emph{\LaTeX pro začátečníky} \cite{rybicka2003} od \emph{Jiřího Rybičku}. 


\section{Inštalácia v OS Windows}

Ak pracujete v \emph{OS Windows}, stiahnite si distribúciu \LaTeX-u s názvom \emph{TeX Live} zo stránky \url{https://www.tug.org/texlive/}. Inštaláciu distribúcie \emph{MikTeX} neodporúčame.


\section{Inštalácia v OS Linux}

Ak používate distribúciu \emph{Fedora 23}, pre používanie šablóny budete potrebovať nainštalovať nasledujúce balíčky:

\begin{minted}{bash}
$ sudo dnf install texlive-bibtopic texlive-cslatex \
                 texlive-collection-latex \
                 texlive-collection-fontsrecommended \
                 texlive-cite latexmk texlive-textcase \
                 texlive-engrec texlive-parskip \
                 texlive-minted \
                 texlive-europecv \
                 texlive-hyphen-english texlive-hyphen-slovak \
                 texlive-titlesec
\end{minted}

\section{Inštalácia v OSX}

TODO: Nakoľko nedisponujeme týmto systémom, ak má niekto z vás skúsenosti s inštaláciou systému \LaTeX na tento operačný systém, ozvite sa ;)


\section{Ďaľšie odporúčané balíčky}
\begin{itemize}
    \item \emph{minted}\footnote{\url{https://www.ctan.org/pkg/minted}} - zvýrazňovač zdrojového kódu (syntax highlighter) pre \LaTeX 
    \item \emph{europecv}\footnote{\url{https://www.ctan.org/pkg/europecv}} - šablóna pre písanie životopisov vo formáte EuroPass 
    \item \emph{PGF/TikZ}\footnote{\url{http://www.ctan.org/pkg/pgf}} - dvojica jazykov, pomocou ktorých je možné vytvárať vektorovú grafiku 
    \item \emph{rotating}\footnote{\url{https://www.ctan.org/pkg/rotating}} - umožňuje otáčať obrázky a tabuľky (spolu s ich popiskami)
\end{itemize}


\section{Generovanie \LaTeX dokumentov}

Pre priame spúšťanie z príkazového riadku odporúčame použiť príkaz {\tt latexmk}, ktorý slúži na zostavenie  \LaTeX dokumentov. Príklad použitia je nasledovný:

\begin{minted}{bash}
$ latexmk -pdf -bibtex -shell-escape thesis
\end{minted}

Ak nechcete spúšťať tento príkaz zakaždým po vykonaní zmien v zdrojových súboroch, pridajte programu {\tt latexmk} prepínač {\tt -pvc}, ktorý zabezpečí ich sledovanie a znovuzostavenie výstupu automaticky:

\begin{minted}{bash}
$ latexmk -pdf -bibtex -shell-escape -pvc thesis
\end{minted}

Ak budete chcieť vyčistiť vygenerované výstupy, stačí nástroj {\tt latexmk} spustiť s prepínačom {\tt -c}:

\begin{minted}{bash}
$ latexmk -c thesis
\end{minted}


\section{Nástroj vlna}

Význam tohto nástroja je zhrnutý v úvode jeho manuálovej stránky:

\begin{quote}
There exists a special Czech and Slovak typographical rule: you cannot leave the non-syllabic preposition on the end of one line and continue writting text on next line. For  example, you  cannot  write  down  the text "v lese" (in a forest) like "v<new-line>lese". The program vlna adds the asciitilde between  such  preposition  and  the next  word and removes the space(s) in this place.  It means, the program converts "v lese" to "v~lese". You  can  use  this  program  as  a  preporcessor  before  TeXing. Moreower,  you  can  set  another  sequence  to  store instead asciitilte (see the -x option).
\end{quote}

Takže ak záverečnú prácu píšete v slovenskom alebo českom jazyku, odporúčame nástroj vlna spustiť minimálne raz pred samotným odovzdaním. Samozrejme ho môžete spustiť vždy, keď svoju prácu budete posielať svojmu školiteľovi alebo konzultantovi na kontrolu.

%
%\include{uvod}
%%
%% !TEX root = ../thesis.tex

\chapter{Formulácia úlohy}

Text záverečnej práce musí obsahovať\/ kapitolu s~formuláciou úlohy resp. úloh riešených v~rámci záverečnej práce. V~tejto časti autor rozvedie spôsob, akým budú riešené úlohy a~tézy formulované v~zadaní práce. Taktiež uvedie prehľad podmienok riešenia.

%%
%\chapter{Analytická časť práce}

% lorem ipsum
\section{Lorem ipsum}
\Blindtext
\blinditemize

\section{Aliquam eu malesuada urna}
\blindtext
\begin{itemize}
    \item v knihe \cite{book} autor prezentuje naozaj odvážne myšlienky
    \item nemenej zujímavé výsledky publikuje ďalší autor v článku \cite{article} 
    \item v konferenčnom príspevku \cite{conference} sú uvedené tiež zauímavé veci
\end{itemize}

Given a set of numbers, there are elementary methods to compute its \acrlong{gcd}, which is abbreviated \acrshort{gcd}. This process is similar to that used for the \acrfull{lcm}.

\subsection{Donec vehicula consequat}
\blindtext

\subsection{Nullam in mauris consectetur}
\blindtext
\blindenumerate

\subsection{Vestibulum tristique elementum varius}
\blindtext

\section{Phasellus id pretium neque}
\Blindtext

%%
%\include{jadroprace}
%%
%% !TEX root = ../thesis.tex

\chapter{Záver}

% lorem ipsum
\Blindtext

%%
%\include{literatura}
%%
%\chapter*{Zoznam príloh}

\addcontentsline{toc}{chapter}{\numberline{}Zoznam príloh}
\thispagestyle{empty}

\begin{description}
	\item[Príloha A] Lorem ipsum text
    \item[Príloha B] CD médium -- záverečná práca v~elektronickej podobe,
    \item[Príloha C] Používateľská príručka
    \item[Príloha D] Systémová príručka
\end{description}


%%
%% !TEX root = ../thesis.tex

\chapter{Karel Language Reference}

\section*{Karel's Primitives}

\begin{itemize}
    \item \verb|void movek()| - Moves \textit{Karel} one intersection forward.
    \item \verb|void turn_left()| - Pivots \textit{Karel} $90$ degrees left.
    \item \verb|void pick_beeper()| - Takes a beeper from the current intersection and puts it in the beeper bag.
    \item \verb|void put_beeper()| - Takes a beeper from the beeper bag and puts it at the current intersection.
    \item \verb|void turn_on(char* path)| - Turns \textit{Karel} on.
    \item \verb|void turn_off()| - Turns \textit{Karel} off.
\end{itemize}


\section*{Karel's Sensors}

\begin{itemize}
    \item \verb|int front_is_clear()| - Returns \texttt{1} if there is no wall directly in front of \textit{Karel}. \texttt{0} if there is.
    \item \verb|int right_is_clear()| - Returns \texttt{1} if there is no wall immediately to \textit{Karel}'s right. \texttt{0} if there is.
    \item \verb|int beepers_present()| - Returns \texttt{1} if \textit{Karel} is standing at an intersection that has a beeper. \texttt{0} otherwise.
    \item \verb|int facing_north()| - Returns \texttt{1} if \textit{Karel} is facing north. \texttt{0} otherwise.
    \item \verb|int beepers_in_bag()| - Returns \texttt{1} if there is at least one beeper in \textit{Karel}'s beeper bag. \texttt{0} if the beeper bag is empty.
\end{itemize}


\section*{Misc Functions}

\begin{itemize}
    \item \verb|void set_step_delay(int)| - Sets delay of one \textit{Karel}'s step in miliseconds.
    \item \verb|loop(int)| - Repeats \textit{Karel}'s instruction in a loop.
\end{itemize}

%%
%\include{prilohab}
%%
%\include{prilohac}
%
% zivotopis autora
\curriculumvitae\protect\label{page:posledna}
Táto časť\/ je nepovinná. Autor tu môže uviesť\/ svoje biografické
údaje, údaje o~záujmoch, účasti na~projektoch, účasti na~súťažiach,
získané ocenenia, zahraničné pobyty na~praxi, domácu prax, publikácie
a~pod.
\kcurriculumvitae

\end{document}
%%
