% !TEX root = ../thesis.tex

\chapter{Záver}
\label{summary}

V tejto bakalárskej práci, bolo prvým krokom analyzovanie a stručný opis platformy Kubernetes, ktorá sa vzhľadom na problematiku tejto práce využíva na riešenie úloh strojového učenia. Čitateľ je oboznámený s výhodami tejto platformy, ktoré ponúka. Za dôležité sa považoval opis a rozdelenie architektúry jednotlivých komponentov, z ktorých je platforma zlozená. Sú často spomínane v ďalších kapitolách, preto je vhodné sa s nimi zoznámiť. Porovnané sú niektoré známe nástroje, ktoré sú dostupné na internete na vytvorenie klastra Kubernetes. Kontajnerizácia je neodmysliteľnou súčasťou správnej funkčnosti tejto platformy. Poukazuje na to, aké má výhody oproti virtualizačným systémom a preto je adekvátna na riešenie daného problému.

Na riešenie problémov strojového učenia je najlepším a zároveň najznámejším systémom Kubeflow, slúžiaci na nasadenie pre platformu Kubernetes, ktorý obsahuje všetko pre obľúbencov a inžinierov strojového učenia. Súčasťou je charakteristika dôležitých časti a pracovných postup pre predstavu o chode tohto systému. Opísane sú postupy na vytvorenie testovacieho prostredia využívajúce tento systém, pre spúšťanie úloh naprogramovaných v jazyku Python. Používateľ je bližšie informovaný o krokoch a postupoch, ktoré sú žiadane a aké požiadavky tento scenár nasadenia vyžaduje od hardvérových až systémových nárokov. V prípade, že scenár podporuje GPU a viacuzlový klaster pri danom scenári je pripojenie strojov alebo podpora grafickej karty.

Na záver sú porovnané a zhodnotené tieto nasadenia s odôvodneným výberom najlepšieho scenára pre splnenie požiadaviek tejto práce. Výsledkom je vytvorený Kubernetes klaster s pripojenými strojmi umožňujúci nasadenie úloh strojového učenia. Obsahuje aj manuál na vytvorenie a pripojenie viacerých počítačov do platformy Kubernetes a tiež manuál na implementáciu a nasadenie Python zdrojových kódov spustiteľných na platforme Kubernetes.