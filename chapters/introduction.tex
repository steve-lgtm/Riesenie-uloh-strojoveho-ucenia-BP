% !TEX root = ../thesis.tex

\chaptermark{Úvod}
\phantomsection
\addcontentsline{toc}{chapter}{Úvod}

\chapter*{Úvod}

Strojové učenie v dnešnej dobe veľmi rýchlo napreduje. Je dôležitou súčasťou rastúcej oblasti vedy o údajoch. Pomocou štatistických metód sú algoritmy trénované na vytváranie klasifikácií alebo predpovedí, ktoré odhaľujú kľúčové poznatky v~rámci rôznych projektov. Tieto poznatky následne riadia rozhodovanie, čo v~ideálnom prípade ovplyvňuje kľúčové metriky. Vyžaduje sa od nich, aby pomáhali pri identifikácii najrelevantnejších otázok a následne údajov, ktoré by na nich odpovedali. Pri experimentoch strojového učenia je niekedy vyžadovaný vyšší výkon. Ak sú prístupné viaceré stroje, ktoré sú rozdelené medzi viacerými používateľmi a na danom počítači je žiadaný vyšší výkon, migrovanie údajov manuálne medzi strojmi by bolo zdĺhavé. Ideálnym riešením je, aby Kubernetes bežal na strojoch a nasadzovali by sa naň úlohy bez toho, aby sa muselo riešiť na ktorom stroji sa spustia.

Riešenie úloh strojového učenia sa stáva bežne implementovaným nástrojom na uľahčenie pracovnej záťaže zamestnancov a vedcov v rôznych oblastiach, od kybernetickej bezpečnosti až po služby zákazníkom. Možným riešením, ktoré môže priniesť výhody, je open-source technológia kontajnerizácie Kubernetes. Umožňuje rýchlu, jednoduchú správu a prehľadnú organizáciu kontajnerových služieb a aplikácií. Táto technológia tiež umožňuje automatizáciu prevádzkových úloh, ako je správa dostupnosti aplikácií a škálovanie. Podporuje grafickú kartu, čo urýchľuje pracovný tok a automatizuje správu kontajnerov aplikácií akcelerovaných grafickou kartou. Tieto nástroje umožňujú využiť ich rýchlosť v rámci kontajnerového pracovného postupu.

Niekedy sú bežiace postupy na Kubernetes komplikované. Tento problém je možné vyriešiť nadstavbou Kubeflow. Pomáha efektívne spúšťať, organizovať a škálovať modely nezávisle od ich závislostí, ako často musia byť aktívne a koľko údajov potrebujú spracovať. Cieľom Kubeflow je uľahčiť inžinierom strojového učenia a vedcom využívať prostriedky pre pracovné zaťaženie strojového učenia. Dokáže efektívne vytvorenie modelu pripraveného na produkčné použitie za veľmi krátky čas.

Existuje veľa distribútorov, firiem, ktoré vyvíjajú tieto platformy a sú rôzne prispôsobované, preto je cieľom tejto práce otestovať jednotlivé spôsoby a možnosti týchto nasadení. Je potrebné zohľadňovať viaceré faktory. Faktory ako sú podpora grafickej karty, možnosť prepojenia viacerých strojov, systémové požiadavky alebo softvérové rámce.

Výber správnej implementácie je dôležitý. Mala by mať čo najmenej problémov, s ktorými sa autor tejto práce stretne a poskytovala čo najlepšie hardvérové využitie.

\clearpage

\section*{Formulácia úlohy}

Hlavnou úlohou tejto práce je vytvoriť testovacie prostredie Kubernetes s možnosťou spúšťania úloh strojového učenia naprogramovaných v jazyku Python. Najprv je vhodné venovať pozornosť analýze tejto platformy. Platforma Kubernetes je na prvý pohlaď náročná, taktiež vďaka rôznym výrazom, ktoré s ňou súvisia. Na mieste je opis týchto komponentov, architektúry a kontajnerizácie, pre lepšie pochopenie tejto platformy pre používateľa. Otestovanie nasadení využitím rôznych nástrojov, ktoré je možné otestovať a vyhodnotiť. Uskutočniť prepojenie viacerých počítačov v klastri a odskúšať funkčnosť podpory grafickej karty. Po testovaní je na rade porovnanie a vyhodnotenie jednotlivých scenárov a výber najlepšieho nasadenia, vhodného na vyriešenie danej problematiky. V poslednom rade je vyhotovenie manuálu na vytvorenie a pripojenie viacerých počítačov do platformy Kubernetes a manuál na implementáciu a nasadenie Python zdrojových kódov spustiteľných na platforme Kubernetes.