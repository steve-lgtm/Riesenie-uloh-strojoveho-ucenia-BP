% !TEX root = ../thesis.tex

\chapter{Systémová príručka}

\section{Funkcia platforiem}

popis

\section{Inštalácia riadiacej roviny Kubernetes}

\subsection*{Systémové požiadavky}

Na inštaláciu platformy Kubernetes ako riadiací uzol je náležitosťou, aby stroj spĺňal minimálne systémové požiadavky:

\begin{itemize}
    \item Operačný systém: \textbf{Ubuntu 18.04 LTS a vyšší}
	\item Počet jadier procesora: \textbf{2}
    \item Veľkosť operačnej pamäte RAM: \textbf{12 GB}
    \item Veľkosť pamäťového úložiska: \textbf{50 GB}
\end{itemize}

\subsection*{Postup inštalácie}
\label{sec:hello}

Inštalácia je vykonávaná v terminály Ubuntu v nasledovnom poradí:
\begin{enumerate}
    \item{\noindent Odporúča sa prejsť do režimu \textbf{root} vykonaním príkazu a zadaním hesla.
\begin{lstlisting}[language=Bash,basicstyle=\footnotesize]
    sudo -s
    \end{lstlisting}}
\item{\noindent Nastavenie pravidiel firewallu, aby bol viditeľný premostený prenos.
\begin{lstlisting}[language=Bash,basicstyle=\footnotesize]
    cat <<EOF | sudo tee /etc/modules-load.d/k8s.conf
    br_netfilter
    EOF

    cat <<EOF | sudo tee /etc/sysctl.d/k8s.conf
    net.bridge.bridge-nf-call-ip6tables = 1
    net.bridge.bridge-nf-call-iptables = 1
    EOF
    sudo sysctl --system
    \end{lstlisting}}
    \item{\noindent Kvôli Kubeadm je potrebné zakázať swap na všetkých uzloch.
\begin{lstlisting}[language=Bash,basicstyle=\footnotesize]
    sudo swapoff -a
    sudo sed -i '/ swap / s/^\(.*\)$/#\1/g' /etc/fstab
    \end{lstlisting}}
    \item{\noindent Nasleduje inštalácia požadovaných balíkov pre Docker.
\begin{lstlisting}[language=Bash,basicstyle=\footnotesize]
    sudo apt-get update -y
    sudo apt-get install -y apt-transport-https ca-certificates \
        curl gnupg lsb-release
    \end{lstlisting}}
    \item{\noindent Pridanie kľúča Docker GPG a apt repozitára.
\begin{lstlisting}[language=Bash,basicstyle=\footnotesize]
    curl -fsSL https://download.docker.com/linux/ubuntu/gpg | \
        sudo gpg --dearmor -o /usr/share/keyrings/\
        docker-archive-keyring.gpg

    echo \
        "deb [arch=amd64 signed-by=/usr/share/keyrings/\
        docker-archive-keyring.gpg] \
        https://download.docker.com/linux/ubuntu \
        $(lsb_release -cs) stable" | \
        sudo tee /etc/apt/sources.list.d/docker.list > /dev/null
\end{lstlisting}}
\item{\noindent Inštalácia samotnej komunitnej verzie Dockera.
\begin{lstlisting}[language=Bash,basicstyle=\footnotesize]
    sudo apt-get update -y
    sudo apt-get install docker-ce docker-ce-cli containerd.io -y
\end{lstlisting}}
\item{\noindent Pridanie konfigurácie Docker daemona pre použitie systemd ako cgroup ovládača.
\begin{lstlisting}[language=Bash,basicstyle=\footnotesize]
    cat <<EOF | sudo tee /etc/docker/daemon.json
    {
      "exec-opts": ["native.cgroupdriver=systemd"],
      "log-driver": "json-file",
      "log-opts": {
        "max-size": "100m"
      },
      "storage-driver": "overlay2"
    }
    EOF
\end{lstlisting}}
\item{\noindent Spustenie a povolenie služby ovládačov.
\begin{lstlisting}[language=Bash,basicstyle=\footnotesize]
    sudo systemctl enable docker
    sudo systemctl daemon-reload
    sudo systemctl restart docker
\end{lstlisting}}
\item{\noindent Dodanie potrebných nástrojov pre Kubeadm.
\begin{lstlisting}[language=Bash,basicstyle=\footnotesize]
    sudo apt-get update
    sudo apt-get install -y apt-transport-https /
        ca-certificates curl
    sudo curl -fsSLo \
        /usr/share/keyrings/kubernetes-archive-keyring.gpg \
        https://packages.cloud.google.com/apt/doc/apt-key.gpg
\end{lstlisting}}
\item{\noindent Pridanie kľúča GPG a repozitára apt pre Kubeadm.
\begin{lstlisting}[language=Bash,basicstyle=\footnotesize]
    echo "deb \
        [signed-by=/usr/share/keyrings/\
        kubernetes-archive-keyring.gpg] \
        https://apt.kubernetes.io/ kubernetes-xenial main" | \
        sudo tee /etc/apt/sources.list.d/kubernetes.list
\end{lstlisting}}
\item{\noindent Inštalácia Kubelet, Kubeadm a kubectl služieb s verziou 1.21.12-00.
\begin{lstlisting}[language=Bash,basicstyle=\footnotesize]
    sudo apt-get update -y
    sudo apt-get install -y kubelet=1.21.12-00 \
    kubectl=1.21.12-00 kubeadm=1.21.12-00
\end{lstlisting}}
\item{\noindent Zablokovanie aktualizácie služieb pred nežiadanou aktualizáciou.
\begin{lstlisting}[language=Bash,basicstyle=\footnotesize]
    sudo apt-mark hold kubelet kubeadm kubectl
\end{lstlisting}}
\item{\noindent Inicializácia klastra využitím nástroja Kubeadm. Namiesto x.x.x.x je\br potrebné zadať IP adresu počítača. Vygeneruje sa token, ktorý je vhodné uchovať pre pripájanie ďalších strojov.
\begin{lstlisting}[language=Bash,basicstyle=\footnotesize]
    IPADDR="x.x.x.x"
    NODENAME=$(hostname -s)
    sudo kubeadm init --apiserver-advertise-address=$IPADDR \
        --apiserver-cert-extra-sans=$IPADDR  \
        --pod-network-cidr=192.168.0.0/16 \
        --node-name $NODENAME --ignore-preflight-errors Swap
\end{lstlisting}}
\item{\noindent Vytvorenie súboru kubeconfig pre službu kubectl na interakciu s klastrom.
\begin{lstlisting}[language=Bash,basicstyle=\footnotesize]
    mkdir -p $HOME/.kube
    sudo cp -i /etc/kubernetes/admin.conf $HOME/.kube/config
    sudo chown $(id -u):$(id -g) $HOME/.kube/config
\end{lstlisting}}
\item{\noindent Overenie funkcionality zobrazením spustených podov.
\begin{lstlisting}[language=Bash,basicstyle=\footnotesize]
    kubectl get po -n kube-system
\end{lstlisting}}
\item{\noindent Ako predvolené nastavenie sa aplikácie naplánujú na riadiacej rovine. Ak je žiadané nasadenie aj na riadiacej rovine nevyhnutné je vykonanie nasledujúceho príkazu.
\begin{lstlisting}[language=Bash,basicstyle=\footnotesize]
    kubectl taint nodes --all node-role.kubernetes.io/master-
\end{lstlisting}}
\item{\noindent Kubeadm neposkytuje žiaden sieťový prvok. Nakonfiguruje sa sieťový doplnok Calico.
\begin{lstlisting}[language=Bash,basicstyle=\footnotesize]
    kubectl apply -f \
        https://docs.projectcalico.org/manifests/calico.yaml
\end{lstlisting}}
\item{\noindent Kubernetes dashboard nie je predvolene nasadený. Pre nasadenie sa použije nasledovný príkaz.
\begin{lstlisting}[language=Bash,basicstyle=\footnotesize]
    kubectl apply -f \
        https://raw.githubusercontent.com/kubernetes/\
        dashboard/v2.5.0/aio/deploy/recommended.yaml
\end{lstlisting}}
\item{\noindent Pri používaní dashboardu sa vyžaduje vytvorenie používateľa, ktoré pozostáva z vytvorenia súborov a nasadenia do klastra.\br
\indent \textbf{admin-user.yaml:}
\begin{lstlisting}[language=Bash,basicstyle=\footnotesize]
    apiVersion: v1
    kind: ServiceAccount
    metadata:
        name: admin-user
        namespace: kubernetes-dashboard
\end{lstlisting}
\indent \textbf{clusterRoleBinding.yaml:}
\begin{lstlisting}[language=Bash,basicstyle=\footnotesize]
    apiVersion: rbac.authorization.k8s.io/v1
    kind: ClusterRoleBinding
    metadata:
        name: admin-user
    roleRef:
        apiGroup: rbac.authorization.k8s.io
        kind: ClusterRole
        name: cluster-admin
    subjects:
        - kind: ServiceAccount
        name: admin-user
        namespace: kubernetes-dashboard
\end{lstlisting}
\indent \textbf{Aplikovanie súborov:}
\begin{lstlisting}[language=Bash,basicstyle=\footnotesize]
    kubectl apply -f admin-user.yaml
    kubectl apply -f clusterRoleBinding.yaml
\end{lstlisting}}
\item{\noindent Vygenerovanie tokenu pre prihlásenie do dashboardu Kubernetes.
\begin{lstlisting}[language=Bash,basicstyle=\footnotesize]
    kubectl -n kubernetes-dashboard get secret $(kubectl -n \
        kubernetes-dashboard get sa/admin-user -o \
        jsonpath="{.secrets[0].name}") -o \
        go-template="{{.data.token | base64decode}}"
\end{lstlisting}}
\item{\noindent Po spustení tohto príkazu je prihlásenie možné zadaním tokenu prostredníctvom prehliadača na adrese\footnote{\url{http://localhost:8001/api/v1/namespaces/kubernetes-dashboard/services/https:kubernetes-dashboard:/proxy/}}.
\begin{lstlisting}[language=Bash,basicstyle=\footnotesize]
    kubectl proxy
\end{lstlisting}}
\item{\noindent Vytvorenie lokálneho trvalého zväzku a nastavenie predvoleného úložiska.
\begin{lstlisting}[language=Bash,basicstyle=\footnotesize]
    kubectl apply -f \
        https://raw.githubusercontent.com/rancher/local-path-\
        provisioner/master/deploy/local-path-storage.yaml
    kubectl patch storageclass local-path -p'\
        {"metadata":{"annotations":{"storageclass.\
        kubernetes.io/is-default-class":"true"}}}'
\end{lstlisting}}
\item{\noindent Overenie predošlého kroku je možné nasledovnými príkazmi.
\begin{lstlisting}[basicstyle=\footnotesize]
    kubectl -n local-path-storage get pod -o wide
    kubectl get sc
\end{lstlisting}}
\end{enumerate}

\section{Pripojenie pracovných strojov}

\subsection*{Systémové požiadavky}

\begin{itemize}
    \item Operačný systém: \textbf{Ubuntu 18.04 LTS a vyšší alebo Windows server 2019 a vyšší}
	\item Počet jadier procesora: \textbf{2}
    \item Veľkosť operačnej pamäte RAM: \textbf{2 GB}
\end{itemize}

\subsection*{Linux/Ubuntu}

Pripojenie pracovného stroja Ubuntu je vykonávané v terminály Ubuntu v nasledovnom poradí:

\begin{enumerate}
\item{\noindent Vykonanie krokov \textbf{1. - 12.} z kapitoly \hyperref[sec:hello]{\textbf{Inštalácia riadiacej roviny Kubernetes}}}
\item{\noindent Vygenerovanie tokenu slúžiaceho na pripojenie stroja do klastra. Získanie tokenu použitím nasledovného príkazu na riadiacej rovine.
\begin{lstlisting}[basicstyle=\footnotesize]
    kubeadm token create --print-join-command
    \end{lstlisting}}
\item{\noindent Vytvorený token z riadiacej roviny sa spustí na stroji, ktorý je potrebné pripojiť. Znázornená je ukážka príkazu.
\begin{lstlisting}[basicstyle=\footnotesize]
    sudo kubeadm join 10.128.0.37:6443 \
        --token j4eice.33vgvgyf5cxw4u8i \
        --discovery-token-ca-cert-hash sha256:37f94469b58bcc\
        8f26a4aa44441fb17196a585b37288f85e22475b00c36f1c61
    \end{lstlisting}}
    \item{\noindent Kontrola úspešného pripojenia sa uskutoční spustením príkazu na riadiacej rovine pre výpis pripojených strojov.
\begin{lstlisting}[language=Bash,basicstyle=\footnotesize]
    kubectl get nodes
    \end{lstlisting}}
\end{enumerate}

\subsection*{Windows server}

Pri pripájaní stroja s operačným systémom Windows je nutné vykonať niektoré kroky na riadiacej rovine:

\begin{enumerate}
\item{\noindent Stiahnutie kontajnerovej sieti Flannel.
\begin{lstlisting}[basicstyle=\footnotesize]
    wget https://raw.githubusercontent.com/coreos/flannel/\
        master/Documentation/kube-flannel.yml
    \end{lstlisting}}
\item{\noindent Editácia net-conf.json sekcie, pridaním VNI a Portu v kube-flannel.yml.
\begin{lstlisting}[basicstyle=\footnotesize]
    net-conf.json: |
        {
            "Network": "10.244.0.0/16",
            "Backend": {
                "Type": "vxlan",
                "VNI": 4096,
                "Port": 4789
            }
        }
    \end{lstlisting}}
\item{\noindent Aplikovanie Flannel editovanej konfigurácie.
\begin{lstlisting}[language=Bash,basicstyle=\footnotesize]
    kubectl apply -f kube-flannel.yml
    \end{lstlisting}}
\item{\noindent Pridanie Flannelu a kube-proxy kompatibilné so systémom Windows s verziou 1.21.12, ktorá sa používa v tejto príručke.
\begin{lstlisting}[language=Bash,basicstyle=\footnotesize]
    curl -L https://github.com/kubernetes-sigs/sig-windows-tools\
        /releases/latest/download/kube-proxy.yml | \
        sed 's/VERSION/v1.21.12/g' | kubectl apply -f -
    kubectl apply -f https://github.com/kubernetes-sigs/sig-win\
        dows-tools/releases/latest/download/flannel-overlay.yml
    \end{lstlisting}}
\end{enumerate}

\noindent Ďalšie kroky sú vykonané v príkazovom riadku PowerShell ako administrátor:

\begin{enumerate}
\item{\noindent Inštalácia funkcie Containers. Po vykonaní tohto kroku sa vyžaduje reštartovanie.
\begin{lstlisting}[basicstyle=\footnotesize]
    Install-WindowsFeature -Name containers
    \end{lstlisting}}
\item{\noindent Nasledovná je inštalácia Dockera pre Windows serveri. Nutný je reštart.
\begin{lstlisting}[basicstyle=\footnotesize]
    Install-Module -Name DockerMsftProvider -Repository \
    PSGallery -Force
    Install-Package -Name docker -ProviderName DockerMsftProvider
    \end{lstlisting}}
\item{\noindent Stiahnutie a nasadenie služieb Wins, Kubelet a Kubeadm s verziou 1.21.12.
\begin{lstlisting}[language=Bash,basicstyle=\footnotesize]
    curl.exe -LO https://raw.githubusercontent.com/kubernetes-si\
        gs/sig-windows-tools/master/kubeadm/scripts/\
        PrepareNode.ps1
    .\PrepareNode.ps1 -KubernetesVersion v1.21.12
    \end{lstlisting}}
\item{\noindent Získanie tokenu použitím nasledovného príkazu na riadiacej rovine.
\begin{lstlisting}[language=Bash,basicstyle=\footnotesize]
    kubeadm token create --print-join-command
    \end{lstlisting}}
\item{\noindent Vytvorený token z riadiacej roviny sa spustí na stroji, ktorý je potrebné pripojiť. Znázornená je ukážka príkazu.
\begin{lstlisting}[basicstyle=\footnotesize]
    sudo kubeadm join 10.128.0.37:6443 \
        --token j4eice.33vgvgyf5cxw4u8i \
        --discovery-token-ca-cert-hash sha256:37f94469b58bcc\
        8f26a4aa44441fb17196a585b37288f85e22475b00c36f1c61
    \end{lstlisting}}
\item{\noindent Kontrola úspešného pripojenia sa uskutoční spustením príkazu na riadiacej rovine pre výpis pripojených strojov.
\begin{lstlisting}[language=Bash,basicstyle=\footnotesize]
    kubectl get nodes
    \end{lstlisting}}
\end{enumerate}

\section{Nasadenie Kubeflow do klastra Kubernetes}

Vykonané sú príkazy prostredníctvom riadiacej roviny v nasledovnom poradí:

\begin{enumerate}
\item{\noindent Pre nasadenie Kubeflow je potrebná služba Kustomize, vyžaduje sa stiahnutie súboru \textbf{kustomize-3.2.0-linux-amd64} z internetovej adresy\footnote{\url{https://github.com/kubernetes-sigs/kustomize/releases/tag/v3.2.0}}. Po stiahnutí sa súbor musí premenovať na \textbf{kustomize}}.
\item{\noindent Zmena práv súboru kustomize a presunutie do priečinka obsahujúce linuxové príkazy.
\begin{lstlisting}[basicstyle=\footnotesize]
    sudo chmod 755 kustomize
    sudo mv kustomize /bin
    \end{lstlisting}}
\item{\noindent Stiahnutie Git repozitára a prechod do priečinku.
\begin{lstlisting}[language=Bash,basicstyle=\footnotesize]
    git clone https://github.com/kubeflow/manifests.git
    cd manifests
    \end{lstlisting}}
\item{\noindent Nasadenie všetkých komponentov prostredníctvom príkazu.
\begin{lstlisting}[language=Bash,basicstyle=\footnotesize]
    while ! kustomize build example | kubectl apply -f -; \
        do echo "Retrying to apply resources"; sleep 10; done
    \end{lstlisting}}
\item{\noindent Po niekoľkých minútach, ak sú všetky pody spustené, je možné sa pripojiť prostredníctvom prehliadača na adrese\footnote{\url{http://localhost:8080}}. Predvolená prihlasovacia emailová adresa je \textbf{user@example} a heslo \textbf{12341234}.}
\end{enumerate}
