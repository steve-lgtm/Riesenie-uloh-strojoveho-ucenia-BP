% Glossary
% ========
%
% A glossary is a list of terms in a particular domain of knowledge with definitions for those terms.
%
% Glossary Definition Exapmle:
% --------
\newglossaryentry{dashboard}
 {
     name={Dashboard},
     description={je grafické používateľské rozhranie, pre interakciu užívateľa s platformou}
}
\newglossaryentry{manifest}
 {
     name={Manifest},
     description={je súbor obsahujúci metadáta pre skupinu sprievodných súborov, ktoré sú súčasťou súboru alebo koherentnej jednotky}
}
\newglossaryentry{pipelines}{
    name={Pipelines},
    description={je platforma na vytváranie a nasadzovanie prenosných, škálovateľných pracovných postupov strojového učenia založených na kontajneroch Docker}
}
\newglossaryentry{snap}{
    name={Snaps},
    description={sú kontajnerové softvérové balíky, ktoré sa jednoducho vytvárajú a inštalujú na operačnom systéme Linux}
}
\newglossaryentry{hypervizor}{
    name={Hypervizor},
    description={je označenie pre jednu z techník virtualizácie hardvéru}
}
\newglossaryentry{Jupyter}{
    name={Jupyter Notebook},
    description={je originálna webová aplikácia na vytváranie a zdieľanie výpočtových dokumentov}
}
\newglossaryentry{katib}{
    name={Katib},
    description={je škálovateľný a rozšíriteľný rámec automatického strojového učenia na Kubernetes}
}
\newglossaryentry{kubectl}{
    name={Kubectl},
    description={je nástroj príkazového riadka na komunikáciu so serverom klastra}
}
\newglossaryentry{kubeadm}{
    name={Kubeadm},
    description={je nástroj používaný na vytváranie klastrov}
}\newglossaryentry{kubelet}{
    name={Kubelet},
    description={predstavuje agenta, ktorý beží na každom uzle v klastri}
}\newglossaryentry{kustomize}{
    name={Kustomize},
    description={umožňuje prispôsobiť nespracované súbory YAML}
}
\newglossaryentry{iptables}{
    name={IPtables},
    description={sa používa na nastavenie, údržbu a kontrolu tabuliek pravidiel filtrovania paketov IP v jadre Linuxu}
}\newglossaryentry{Docker}{
    name={Docker},
    description={jeho cieľom je poskytnúť jednotné rozhranie pre izoláciu aplikácií do kontajnerov v prostredí Linuxu i Windows}
}\newglossaryentry{hyperv}{
    name={Hyper-V},
    description={je hypervízorovo stavaný serverový systém pre Windows}
}
%
%
% Usage:
% ------
% Because most of you will write thesis in Slovak, the best recommendation is to use \glslink command as follows:
% Na konceptuálnom \glslink{class diagram}{diagrame tried}, ktorý sa nachádza na obrázku ...
%
% For more information see:
% -------------------------
% * https://www.sharelatex.com/learn/Glossaries
% * https://en.wikibooks.org/wiki/LaTeX/Glossary
%
